% REQUIRED PACKAGES    ---------------------------------------------------------------------------------
%
%
% texlive-latex-base
% texlive-latex-recommended
% texlive-latex-extra
% texlive-fonts-recommended
% texlive-fonts-extra
% texmaker
% texmaker-data
% cm-super
%
%
%
%%  FORMAT     ---------------------------------------------------------------------------------
\documentclass[11pt, a4paper, sans, blue]{moderncv}
\moderncvstyle{casual} % CV theme: 'casual', 'classic', 'oldstyle', 'banking'
\moderncvcolor{green}

%%  INFO       ---------------------------------------------------------------------------------
\firstname{Michael}
\familyname{Jahn}
\title{Curriculum vitae}
\address{Street name, number}{ZIP code City, Country}
\mobile{+49\,123\,456789}
%\phone{}
\email{jahn$@$email.de}
\photo[0.25\textwidth]{../figures/profile_2020.png}



\begin{document}

%%  OPTIONAL COVERLETTER    ---------------------------------------------------------------------
% \input{coverletter.tex}

\newpage

%%  ENTRIES     ---------------------------------------------------------------------------------

\maketitle

\setcounter{page}{1}


\section{Persönliches}
\cvitem{Name}{Dr. rer. nat. Michael Jahn}
\cvitem{Geburtsdatum}{31. Dezember 1985}
\cvitem{Geboren in}{Dresden}
\cvitem{Addresse}{Strasse, PLZ, Ort}
\cvitem{Mobil}{+49\,123\,456\,789}
\cvitem{Email}{jahn@email.de}


\section{Aktuelle Anstellung}
\cvitem{}{\textbf{Wissenschaftler}, Bioinformatics Platform, Arbeitsgruppe von Emmanuelle Charpentier, Max-Planck-Unit for the Science of Pathogens (MPUSP), Berlin.}


\section{Ausbildung}
\cventry{10. Juli 2015}{Verleihung des \textit{Doctor rerum naturalium}}{}{ der Universität Leipzig, mit Auszeichnung (\textit{summa cum laude})}{}{}
\cventry{2011--2015}{Doktorand}{}{in der Gruppe von Prof. Susann Müller, Abteilung Umweltmikrobiologie, Helmholtz-Zentrum für Umweltforschung (UFZ), Leipzig. \newline
Doktorarbeit: "Characterization of population heterogeneity in a model biotechnological process using \textit{Pseudomonas putida}"}{}{}
\cventry{2005--2011}{Diplomstudium Biologie}{}{Technische Universität Dresden (TUD), Hauptfächer: Genetik, Biochemie, Immunologie. \newline
	Diplomarbeit: "Dynamic mating pheromone gradients for induction of mating projection and fluorescence in yeast", AG Prof. Gerhard Rödel, Note 1.2}{}{}
\cventry{1996--2004}{Gymnasium}{}{Romain-Rolland-Gymnasium, Dresden}{Abitur mit Note 1.6}{}{}
%\cventry{1992--1996}{Grundschule}{}{Dresden}{}{}


\section{Arbeitserfahrung}
\cventry{8/2022--now}{Wissenschaftler}{}{Bioinformatics Platform, Arbeitsgruppe von Emmanuelle Charpentier, Max-Planck-Unit for the Science of Pathogens (MPUSP), Berlin}{}{}
\cventry{05/2016--05/2022}{Wissenschaftler}{}{in der Arbeitsgruppe von Prof. Paul Hudson, Science For Life Laboratory -- Königlich Technische Hochschule (KTH), Stockholm. Systembiologie und Data Science zur Untersuchung des Metabolismus autotropher Bakterien.}{}{}
\cventry{Aug-Dez 2009}{Erasmus-Praktikum}{}{Center of Excellence in Evolutionary Genetics and Physiology, AG Prof. Mikko Nikinmaa, Abteilung Biologie, Universität Turku, Finnland}{}{}
\cventry{2008--2009}{Studentische Hilfskraft}{}{Institut für Genetik}{AG Prof. Gerhard Rödel, Abteilung Biologie, Technische Universität Dresden}{}
\cventry{2007--2008}{Studentische Hilfskraft}{}{Mitteldeutscher Praxisverbund Humangenetik, Dresden}{}{}
\cventry{2004--2005}{Freiwilliges Ökologisches Jahr}{}{Nationalpark Sächsische Schweiz}{Abteilung Umweltbildung}{}


\section{Elternzeit}
\cvitem{11/2019--3/2020}{Elternzeit mit zweitem Kind Sophie}
\cvitem{9/2015--4/2016}{Elternzeit mit erstem Kind Simon}


\section{Auszeichungen}
\cvitem{25 Nov 2015}{\textbf{PhD-Award} 2015 des Helmholtz-Zentrums für Umweltforschung, Leipzig. Jährlich verliehen für ausgezeichnete Dissertationen, dotiert mit 1,000 €.}


\section{Eingeworbene Drittmittel}
\cvitem{11 Nov 2019}{Projekt: \textbf{Investigation of protein resource allocation in the model CO\textsubscript{2} fixing bacterium \textit{Ralstonia eutropha}}.
Zuwendungsgeber: Schwedischer Wissenschaftsrat, FORMAS. Projektnummer: 2019-01491. Projektleiter: Dr. Michael Jahn. Förderperiode 01. 01. 2020 -- 31. 12. 2021. Fördermittel: 2,000,000 SEK (200,000 Euro).}

%\section{Beantragte Drittmittel}
%\cvitem{26. Jan 2016}{Projekt \textbf{Entwicklung von Einzelzell-Analyseverfahren zur Optimierung der N-Glykosylierung in \textit{Saccharomyces cerevisiae}}. \newline
%Zuwendungsgeber: Deutsche Forschungsgemeinschaft (DFG). Auslandsstipendium, 2 Jahre. Projektnummer: JA 2529/1-1. Positiv evaluiert aber zurückgezogen wegen Arbeitsangebot in Stockholm.}
%
%\cvitem{1. Jun 2015}{Projekt \textbf{Analyse der Variabilität von Plasmid- und Chromosomenzahl in Bakterien}. Koautor/Mentor für Doktorarbeit.\newline
%Zuwendungsgeber: Deutsche Bundesstiftung Umwelt (DBU). Abgelehnt in zweiter Evaluationsrunde.}

\section{Lehre und Betreuung}
\cvitem{Lehre}{\textbf{Vorlesung für den Kurs KE2130 "Renewable fuels"}, an der KTH Stockholm. Wintersemester 2020/2021 und 2021/2022. Thema: Systembiologie, Metabolic engineering, Biokatalyse, Modellierung, Flux balance analysis.}

\cvitem{}{\textbf{Vorlesung für den Kurs "Einführung in die Programmiersprache R"} und deren Anwendung für die Durchflusszytometrie, am Helmholtz-Zentrum für Umweltforschung, Leipzig, 2015.}

\cvitem{Doktoranden-vertretung}{Ehrenamtliches Mitglied der Doktorandenvertretung und des Umweltrates des Helmholtz-Zentrums für Umweltforschung, Leipzig, 2011-2015. Aufgaben: Analyse und Steigerung der Umweltverträglichkeit des UFZ, Beratung und Mentoring von beginnenden Doktoranden, Organisation von Events.}

\cvitem{Betreuung}{Betreuung von Bachelor- und Masterstudenten während Doktoranden- und Postdoc-Tätigkeit%: \newline
%\listitemsymbol Alexander Mattausch, Praktikant Biotechnologie, Universität Heidelberg, 2019\newline
%\listitemsymbol Timothy Bergmann, Masterstudent Biotechnologie, KTH Stockholm, 2019\newline
%\listitemsymbol Raquel Perruca, Masterstudentin, Technische Universität Lund, 2018\newline
%\listitemsymbol Carsten Vorpahl, Masterstudent, Universität Leipzig, 2015\newline
%\listitemsymbol Dominique Türkowsky, Masterstudentin, Universität Leipzig, 2014\newline
}

\section{Qualifikationen und Expertise}

\cvitem{Methoden}{
Analytisches Denken, Verstehen komplexer Zusammenhänge, \newline
Projektmanagement, Betreuung, Mitarbeiter-Führung, \newline
Wissenschaftliches Schreiben für Publikationen, Proposals, Reports, \newline
Design experimenteller Strategien zur Problemlösung, \newline
Explorative Analyse und Visualisierung von komplexen (biologischen) Daten, \newline
Erstellung von Pipelines zur automatisierten Datenverarbeitung, \newline
Anwendung von Clustering und Machine Learning Algorithmen, \newline
Expertise in Hochdurchsatzverfahren (Next generation sequencing, Massenspektrometrie), \newline
% Expertise in Next generation sequencing (Illumina), \newline
% Expertise in Massenspektrometrie, Proteomik, \newline
% Expertise in Flow-Zytometry, Zellsortierung, und Imaging \newline
% Expertise in Bioreaktorkultivierung (Chemostat, Turbidostat), \newline
% Erfahrung in molekularbiologischen Standardmethoden.
}

\cvitem{Programmier-sprachen}{R inklusive Shiny (Experte), Python (fortgeschritten), Linux Bash, Markdown, Latex (gelegentliche Nutzung).}

\cvitem{Software-Entwicklung}{
\link[WeightedTreemaps]{https://github.com/m-jahn/WeightedTreemaps},
\link[ShinyTreemaps]{https://github.com/m-jahn/ShinyTreemaps} --
Visualisierung von Genexpression mit Treemaps, CRAN package.\newline
\link[ShinyProt]{https://m-jahn.shinyapps.io/ShinyProt/},
\link[ShinyLib]{https://m-jahn.shinyapps.io/ShinyLib/},
\link[ShinyMC]{https://m-jahn.shinyapps.io/ShinyMC/} --
Interaktive Apps zur Visualisierung von Proteomikdaten, Gen-Knockout-Bibliotheken, und Überwachen von Bioreaktoren.\newline
\link[fluctuator]{https://github.com/m-jahn/fluctuator} --
R Paket zur Visualisierung von metabolic flux Daten.\newline
\link[lattice-tools]{https://github.com/m-jahn/lattice-tools} --
Erweiterung des R lattice Pakets.\newline
\link[snakemake-crispr-guides]{https://github.com/m-jahn/snakemake-crispr-guides},
\link[nf-core-crispriscreen]{https://github.com/MPUSP/nf-core-crispriscreen} --
Pipelines für das Design und die Datenauswerting von von CRISPR(i) libraries.
}

\cvitem{Modellierung}{
Resource allocation-Modell für \textit{Synechocystis} sp., für \link[GAMS]{https://github.com/m-jahn/cell-economy-model-legacy} und \link[Python]{https://github.com/m-jahn/cell-economy-model}.\newline
%
\link[Genome scale metabolic model]{https://github.com/m-jahn/genome-scale-models} (Python) für \textit{Cupriavidus necator}.\newline
%
Genome scale \link[resource balance analysis (RBA)]{https://github.com/m-jahn/genome-scale-models}-Modell for \textit{Cupriavidus necator} (Python).
}

\cvitem{Weiterbildung}{Anerkannter Projektleiter und Beauftragter für die biologische Sicherheit (gemäß \S ~14,15 GenTSV), 2015.}
% \newline
% Weiterbildung Wissenschaftliches Schreiben und Antragstellung, 2-wöchiger Kurs, 2013.}

%\cvitem{Modellorganismen}{Expertise in Kultivierung und Molekularbiologie von Mikroorganismen (z.B. Cyanobakterien, \textit{Cupriavidus necator}, \textit{Pseudomonas putida}, \textit{E.\,coli}, \textit{S.\,cerevisiae}.}

\section{Peer review und Outreach}
\cvitem{Journal Editor}{Review Editor für Frontiers in Bioengineering and Biotechnology}
\cvitem{Peer review}{Aktiver Gutachter für verschiedene wissenschaftliche Journale, z.B. Microbial Cell Factories, Molecular Systems Biology, Photosynthesis Research, und andere. Eine vollständige Liste ist verfügbar bei \link[Publons]{https://publons.com/researcher/1192906/michael-jahn/}
oder \link[ORCID]{https://orcid.org/0000-0002-3913-153X}.}
%
\cvitem{Outreach}{Alle Software, Pipelines und Datensätze sind verfügbar bei \link[Github]{https://github.com/m-jahn/} oder spezialisierten Datenbanken. Forschungstätigkeiten werden kommuniziert mittels \link[Homepage]{https://m-jahn.github.io/}, \link[ResearchGate]{https://www.researchgate.net/profile/Michael_Jahn}, und \link[Twitter]{https://twitter.com/mich_jahn}.}


\section{Sprachen}
\cvitem{Deutsch}{Muttersprache}
\cvitem{Englisch}{Fliessend in Wort und Schrift, CEF level C1}
\cvitem{Schwedisch}{Fliessend in Wort und Schrift, CEF level B2}{}

%\newpage


% List of publications -----------------------------
\section{Publikationen}
%
% template:
%\cvitem{\listitemsymbol}{\textbf{Jahn M}, Name N. \textit{title}. Journal, \textbf{year}. \link[Link]{}.}

\cvitem{\listitemsymbol}{\underline{Miao R, \textbf{Jahn M}}, Shabestary K, Peltier G, Hudson EP. \textit{CRISPR interference screens reveal growth–robustness tradeoffs in Synechocystis sp. PCC 6803 across growth conditions}. The Plant Cell, \textbf{2023}.
\link[Link]{https://doi.org/10.1093/plcell/koad208}.}

\cvitem{\listitemsymbol}{Grätz L, Kowalski-Jahn M, Scharf MM, Kozielewicz P, \textbf{Jahn M}, Bous J, Lambert NA, Gloriam DE, Schulte G. \textit{Pathway selectivity in Frizzleds is achieved by conserved micro-switches defining pathway-determining, active conformations}. Nature Communications, \textbf{2023}.
\link[Link]{https://doi.org/10.1038/s41467-023-40213-0}.}

\cvitem{\listitemsymbol}{Janasch M, Crang N, Asplund-Samuelsson J, Sporre E, Bruch M, Gynnå A, \textbf{Jahn M}, Hudson EP. \textit{Thermodynamic limitations of PHB production from formate and fructose in Cupriavidus necator}.  Metabolic engineering, \textbf{2022}.
\link[Link]{https://doi.org/10.1016/j.ymben.2022.08.005}.}

\cvitem{\listitemsymbol}{\textbf{Jahn M}, Crang N, Janasch M, Hober A, Forsström B, Kimler K, Mattausch A, Chen Q, Asplund-Samuelsson J, Hudson EP.  \textit{Protein allocation and utilization in the versatile chemolithoautotroph Cupriavidus necator}, eLife, 10, \textbf{2021}.
\link[Link]{https://elifesciences.org/articles/69019}.}

\cvitem{\listitemsymbol}{Karlsen J, Asplund-Samuelsson J, \textbf{Jahn M}, Vitay D, Hudson EP. \textit{Slow Protein Turnover Explains Limited Protein-Level Response to Diurnal Transcriptional Oscillations in Cyanobacteria}. Frontiers in Microbiology, 12, 820, \textbf{2021}.
\link[Link]{https://doi.org/10.3389/fmicb.2021.657379}.}

\cvitem{\listitemsymbol}{Yao L, Shabestary K, Björk SM, Asplund-Samuelsson J, Joensson HN, \textbf{Jahn M}, Hudson EP. \textit{Pooled CRISPRi screening of the cyanobacterium Synechocystis sp PCC 6803 for enhanced industrial phenotypes}. Nature Communications, \textbf{2020}.
\link[Link]{https://www.nature.com/articles/s41467-020-15491-7}.}

\cvitem{\listitemsymbol}{Karlsen J, Asplund-Samuelsson J, Thomas Q, \textbf{Jahn M}, Hudson EP. \textit{Ribosome Profiling of Synechocystis Reveals Altered Ribosome Allocation at Carbon Starvation}. MSystems 3, e00126-18, \textbf{2018}.
\link[Link]{https://msystems.asm.org/content/3/5/e00126-18}.}

\cvitem{\listitemsymbol}{\textbf{Jahn M}, Vialas V, Karlsen J, Maddalo G, Edfors F, Forsström B, Uhlén M, Käll L, Hudson EP. \textit{Growth of Cyanobacteria Is Constrained by the Abundance of Light and Carbon Assimilation Proteins}. Cell Reports 25, 478–486.e8., \textbf{2018}.
\link[Link]{https://linkinghub.elsevier.com/retrieve/pii/S2211124718314852}.}

\cvitem{\listitemsymbol}{Shabestary K, Anfelt J, Ljungqvist E, \textbf{Jahn M}, Yao L, Hudson EP. \textit{Targeted Repression of Essential Genes To Arrest Growth and Increase Carbon Partitioning and Biofuel Titers in Cyanobacteria}. ACS Synthetic Biology, 7, 1669–1675, \textbf{2018}.
\link[Link]{https://pubs.acs.org/doi/10.1021/acssynbio.8b00056}.}

\cvitem{\listitemsymbol}{\textbf{Jahn M}, Vorpahl C, Hübschmann T, Harms H, Müller S. \textit{Copy number variability of expression plasmids determined by cell sorting and Droplet Digital PCR}.  Microbial Cell Factories, \textbf{2016}.
\link[Link]{http://microbialcellfactories.biomedcentral.com/articles/10.1186/s12934-016-0610-8}.}

\cvitem{\listitemsymbol}{Lindmeyer M, \textbf{Jahn M}, Vorpahl C, Müller S, Schmid A, Bühler B. \textit{Variability in subpopulation formation propagates into biocatalytic variability of engineered Pseudomonas putida strains}.  Frontiers in microbiology 6, \textbf{2015}.
\link[Link]{http://journal.frontiersin.org/article/10.3389/fmicb.2015.01042}.}

\cvitem{\listitemsymbol}{Lieder S, \textbf{Jahn M}, Koepff J, Müller S, Takors Ralf. \textit{Environmental stress speeds up DNA replication in Pseudomonas putida in chemostat cultivations}.  Biotechnology journal, \textbf{2015}.
\link[Link]{http://doi.wiley.com/10.1002/biot.201500059}.}

\cvitem{\listitemsymbol}{\textbf{Jahn M}, Günther S, Müller S. \textit{Non-random distribution of macromolecules as driving forces for phenotypic variation}. Current Opinion in Microbiology. 25:49-55, \textbf{2015}.
\link[Link]{https://doi.org/10.1016/j.mib.2015.04.005}.}

\cvitem{\listitemsymbol}{Rödiger S, Burdukiewicz M, Blagodatskikh K, \textbf{Jahn M}, Schierack P. \textit{R as an environment for reproducible analysis of DNA amplification experiments}. R Journal 7/1:127-150, \textbf{2015}.
\link[Link]{https://journal.r-project.org/archive/2015/RJ-2015-011/RJ-2015-011.pdf}.}

\cvitem{\listitemsymbol}{\textbf{Jahn M}, Vorpahl C, Türkowsky D, Lindmeyer M, Bühler B, Harms H, Müller S. \textit{Accurate determination of plasmid copy number of flow-sorted cells using droplet digital PCR}. Analytical Chemistry 86:5969-76, \textbf{2014}.
\link[Link]{https://pubs.acs.org/doi/10.1021/ac501118v}.}

\cvitem{\listitemsymbol}{Lieder S, \textbf{Jahn M}, Seifert J, von Bergen M, Müller S, Takors R. \textit{Subpopulation-proteomics reveal growth rate, but not cell cycling, as a major impact on protein composition in} Pseudomonas putida \textit{KT2440}. AMB Express 4:71, \textbf{2014}.
\link[Link]{http://www.amb-express.com/content/4/1/71}.}

\cvitem{\listitemsymbol}{\textbf{Jahn M}, Seifert J, von Bergen M, Schmid A, Bühler B, Müller S. \textit{Subpopulation-proteomics in prokaryotic populations}. Current Opinion in Biotechnology 24:79-87, \textbf{2013}.
\link[Link]{https://linkinghub.elsevier.com/retrieve/pii/S0958166912001723}.}

\cvitem{\listitemsymbol}{\textbf{Jahn M}, Seifert J, Hübschmann T, von Bergen M, Harms H, Müller S. \textit{Comparison of preservation methods for bacterial cells in cytomics and proteomics}. Journal Of Integrated Omics 3:1-9, \textbf{2013}.
\link[Link]{https://www.jiomics.com/index.php/jiomics/article/view/77}.}

\cvitem{\listitemsymbol}{\textbf{Jahn M}, Mölle A, Rödel G, Ostermann K. \textit{Temporal and spatial properties of a yeast multi-cellular amplification system based on signal molecule diffusion}. Sensors 13:14511-22, \textbf{2013}.
\link[Link]{https://www.mdpi.com/1424-8220/13/11/14511}.}

%\section{Konferenzbeiträge}
%
%\cvitem{Sep 07--09, 2020}{\textbf{11EWBC} - 11th European workshop on the biology of cyanobacteria, Porto, Portugal (virtuell). Vortrag.}
%
%\cvitem{May 22--24, 2019}{\textbf{NPC 14} - Nordic congress on photosynthesis, Turku, Finland. Kurzvortrag.}
%
%\cvitem{Oct 24--26, 2018}{\textbf{4th Applied Synthetic Biology meeting} - Toulouse, France. Vortrag.}
%
%\cvitem{Jul 20--22, 2015}{\textbf{Single Cell VI} -- 6\textsuperscript{th} International Conference on Analysis of Microbial Cells at the Single Cell Level, Retz (Austria). Vortrag.}
%
%\cvitem{Oct 15--17, 2014}{\textbf{DGFZ} -- 24\textsuperscript{th} Annual Conference of the German Society for Cytometry, Dresden. Vortrag.}
%
%\cvitem{Jul 13--16, 2014	}{\textbf{ECB16} -- 16\textsuperscript{th} European Congress on Biotechnology, Edinburgh. Vortrag.}
%
%\cvitem{May 26--28, 2014}{\textbf{DECHEMA} -- Biomaterials Made in Bioreactors, Dresden. Poster.}
%
%\cvitem{Mar 06--08, 2013	}{\textbf{RPP7} -- 7\textsuperscript{th} European Conference on Recombinant Protein Production, Ulm. Vortrag.}
%
%\cvitem{Jul 21--25, 2013	}{\textbf{FEMS} -- 5\textsuperscript{th} Congress of European Microbiologists, Leipzig. Poster.}
%
%\cvitem{Oct 10--12, 2012}{\textbf{DGFZ} -- 22\textsuperscript{nd} Annual Conference of the German Society for Cytometry, Bonn. Vortrag.}
%
%\cvitem{Jun 23--27, 2012}{\textbf{CYTO 2012} -- 27\textsuperscript{th} Congress of the International Society for Advancement of Cytometry, Leipzig. Vortrag.}


\end{document}
